%layout: post
%title: Transformation
%date: 2025-08-23 11:12:00-0400
%description: Arithmetic circuits for Root transformed polynomials
%tags: transforming roots
%categories: ACT
%related_posts: false
%
\documentclass[a4paper,11pt,american]{article}
%%%%%%%%%%%%%%%%%%%%%%%%%%%%%%%%%%%%%%%%%%%%%%%%%%%%%%
% === Geometry and Fonts ===
%%%%%%%%%%%%%%%%%%%%%%%%%%%%%%%%%%%%%%%%%%%%%%%%%%%%%%
\usepackage{geometry}
\geometry{verbose,tmargin=1in,bmargin=1in,lmargin=1in,rmargin=1in}
\usepackage{mathptmx} % Times font
\usepackage{comment} % Times font

%%%%%%%%%%%%%%%%%%%%%%%%%%%%%%%%%%%%%%%%%%%%%%%%%%%%%%
% === Math Packages ===
%%%%%%%%%%%%%%%%%%%%%%%%%%%%%%%%%%%%%%%%%%%%%%%%%%%%%%
\usepackage{amsmath,amssymb,amsthm}
\usepackage{float}
\usepackage{complexity}
\usepackage{tikz}
\usepackage{extarrows}
\usepackage{thmtools} 
\usepackage{thm-restate}
%%%%%%%%%%%%%%%%%%%%%%%%%%%%%%%%%%%%%%%%%%%%%%%%%%%%%%
% === Lists and To-Dos ===
%%%%%%%%%%%%%%%%%%%%%%%%%%%%%%%%%%%%%%%%%%%%%%%%%%%%%%
\usepackage{enumitem}
\setlist{nosep} 
\setlist[description]{font=\normalfont\space}

%%%%%%%%%%%%%%%%%%%%%%%%%%%%%%%%%%%%%%%%%%%%%%%%%%%%%%
% === Colors and Hyperlinks ===
%%%%%%%%%%%%%%%%%%%%%%%%%%%%%%%%%%%%%%%%%%%%%%%%%%%%%%
\usepackage{silence}
\WarningFilter{hyperref}{Token not allowed in a PDF string}
\usepackage[dvipsnames]{xcolor} % enables BrickRed, NavyBlue, etc.

\usepackage[colorlinks]{hyperref}
\usepackage[style=alphabetic]{biblatex}

%%%%%%%%%%%%%%%%%%%%%%%%%%%%%%%%%%%%%%%%%%%%%%%%%%%%%%
% === Theorem Environments ===
%%%%%%%%%%%%%%%%%%%%%%%%%%%%%%%%%%%%%%%%%%%%%%%%%%%%%%
\newtheorem{theorem}{Theorem}[section]
\newtheorem{lemma}{Lemma}[section]
\newtheorem{proposition}{Proposition}[section]
\newtheorem{corollary}{Corollary}[section]
\newtheorem{fact}{Fact}[section]
\newtheorem{problem}{Problem}[section]
\newtheorem{claim}{Claim}[section]
\newtheorem{conjecture}{Conjecture}[section]

\theoremstyle{definition}
\newtheorem{definition}{Definition}[section]
\newtheorem{example}{Example}[section]

\theoremstyle{remark}
\newtheorem{remark}{Remark}[section]

%%%%%%%%%%%%%%%%%%%%%%%%%%%%%%%%%%%%%%%%%%%%%%%%%%%%%%
% === Bibliography ===
%%%%%%%%%%%%%%%%%%%%%%%%%%%%%%%%%%%%%%%%%%%%%%%%%%%%%%
\addbibresource{Bibliography.bib}



\title{Tranformation of Polynomial Roots}

%%%%%%%%%%%%%%%%%%%%%%%%%%%%%%%%%%%%%%%%%%%%%%%%%%%%%%
% === Document Body ===
%%%%%%%%%%%%%%%%%%%%%%%%%%%%%%%%%%%%%%%%%%%%%%%%%%%%%%
\begin{document}

\section{Introduction to Waring Rank and Border Waring Rank}

The concept of Waring rank and its border counterpart are fundamental in algebraic geometry, particularly in the study of tensors and their decompositions.
These concepts are crucial for understanding how homogeneous polynomials can be expressed as sums of powers of linear forms.
\subsection{Waring Rank}
The \textbf{Waring rank} of a homogeneous polynomial $f$ of degree $d$, denoted $\WR(f)$, is the smallest integer $r$ such that $f$ can be expressed as a sum of $d$-th powers of $r$ linear forms. Here, we work with $n+1$ variables $x_0, x_1, \dots, x_n$ over $\mathbb{C}$, and consider homogeneous polynomials of degree $d$ in these variables.

The \textbf{Waring rank} of a homogeneous polynomial $f \in \mathbb{C}[x_0, x_1, \dots, x_n]$ of degree $d$, denoted $\WR(f)$, is the smallest integer $r$ such that $f$ can be written as a sum of $d$-th powers of $r$ linear forms:
\[
\WR(f) = \min \left\{ r \in \bbN \mid f = \sum_{i=1}^{r} \ell_i(x_0, x_1, \dots, x_n)^d,\ \ell_i \text{ linear forms} \right\}
\]

\subsection{Border Waring Rank}
The \textbf{border Waring rank} of a homogeneous polynomial $f$, denoted $\BWR(f)$, is a generalization that accounts for limits of polynomials with low rank. It is defined as the smallest integer $r$ such that $f$ lies in the closure of the set of polynomials with Waring rank at most $r$. This means $f$ can be expressed as a limit of a sequence of polynomials, each having Waring rank at most $r$:
\[
f = \lim_{\varepsilon \to 0} \sum_{i=1}^{r} \ell_i(\varepsilon)^d
\]
where $\ell_i(\varepsilon)$ are linear forms whose coefficients depend rationally on $\varepsilon$.

Another equivalent definition of the border Waring rank involves the expansion of $f$ as the coefficient of the lowest degree term of $\varepsilon$ as follows:
\[
f = \text{Coeff}_{\varepsilon^k} \left( \sum_{i=1}^r \ell_i^d \right)
\]
Here $\ell_i$ are linear forms in $\bbC[\varepsilon][\lstp{x}{n}[0]]$ and  $k$ is the minimal power of $\varepsilon$ in the right hand side of above equation. For example, for $f = x^{d-1}y$, we have
\[
dx^{d-1}y = \text{Coeff}_{\varepsilon} \paren{ (x+\varepsilon y)^d - x^d }
\]
which shows that $\BWR(x^{d-1}y) = 2$, even though $\WR(x^{d-1}y) = d$.

\section{Waring Rank of Monomials}

For monomials, the study of their Waring  has yielded precise formulas that characterize these ranks in terms of the exponents of the monomial.
Let $M = x_0^{a_0} x_1^{a_1} \cdots x_n^{a_n}$ be a monomial of degree $d = \sum_{i=0}^{n} a_i$. We can assume, without loss of generality, that the exponents are ordered as $a_0 \leq a_1 \leq \dots \leq a_n$.

\subsection{Waring Rank of Monomials}
A significant result concerning the Waring rank of a monomial is:
\begin{theorem}[\cite{CarliniCatalisanoGeramita2012,BuczynskaBuczynskiTeitler2014}]\label{thm:tight-bound-upper-bound-monomial-wr}
The Waring rank of a monomial $x_0^{a_0} x_1^{a_1} \cdots x_n^{a_n}$  is given by
\[
\WR(x_0^{a_0} x_1^{a_1} \cdots x_n^{a_n}) = \prod_{i=1}^{n} (a_i + 1)
\]
assuming $a_0 \leq a_1 \leq \dots \leq a_n$.
\end{theorem}

\begin{lemma}[Orthogonality of roots of unity]\label{lem:orthogonality}
Let \( r \in \mathbb{N} \), and \(\zeta \in \mathbb{C}\) be a primitive \(r\)-th root of unity, i.e., \(\zeta^r = 1\) and \(\zeta^k \neq 1\) for \(1 \le k < r\). Then for any integers \(m, k\),
\[
 \sum_{j=0}^{r-1} \zeta^{j(m-k)} = 
\begin{cases}
r & \text{if } m \equiv k \pmod{r}, \\
0 & \text{otherwise}.
\end{cases}
\]
\end{lemma}

\begin{proof}
Set \(\omega \eqdef \zeta^{m-k}\). Then
\[
S \eqdef\sum_{j=0}^{r-1} \zeta^{j(m-k)} = \sum_{j=0}^{r-1} \omega^j.
\]

If \(m \equiv k \pmod{r}\), then \(\omega = 1\) and
\[
S = \sum_{j=0}^{r-1} 1 = r.
\]

Otherwise, \(\omega \neq 1\) and since \(\omega^r = (\zeta^r)^{m-k} = 1\), 
\[
S = \frac{1 - \omega^r}{1 - \omega} = \frac{1 - 1}{1 - \omega} = 0.
\]

\end{proof}

Now we prove \cref{thm:tight-bound-upper-bound-monomial-wr}.

\printbibliography
    

\end{document}
